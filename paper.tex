\documentclass[11pt]{article}

\usepackage{amsmath}
\usepackage{amssymb}
\usepackage{amsthm}
\usepackage{agda} % this should come after ams*
\usepackage{catchfilebetweentags}
\usepackage{fontspec}
\usepackage{geometry}
\usepackage{graphicx}
\usepackage{hyperref}
\usepackage{ifdraft}
\usepackage{ifthen}
\usepackage{lualatex-math}
%\usepackage{mathtools}
\usepackage{proof}
\usepackage{scalerel}
\usepackage{setspace}
\usepackage{stmaryrd}
\usepackage{supertabular}
\usepackage{tikz-cd}
\usepackage{unicode-math}
\usepackage{url}
\usepackage[usenames,dvipsnames,svgnames,table]{xcolor}
\usepackage{xunicode}

\definecolor{dark-gray}{gray}{0.20}

% fonts
\usepackage[
  activate={true,nocompatibility},
  tracking=true
]{microtype}
\setmainfont{Linux Libertine O}
\setmathfont{Asana Math}

% make lualatex play nice with agda
\makeatletter
\ifluatex%
   \newwrite\CATCHPATCHOUT%
\fi
\def\CatchFBT@Fin@l#1[#2]{%
   \begingroup%
      \endlinechar\m@ne\makeatletter #2%
      \ifluatex%
         \immediate\openout\CATCHPATCHOUT=\jobname.catchpatch%
         \immediate\write\CATCHPATCHOUT{\CatchFBT@tok{\the\CatchFBT@tok}}%
         \immediate\closeout\CATCHPATCHOUT%
         \@@input\jobname.catchpatch%
      \else%
      \scantokens\expandafter{%
         \expandafter\CatchFBT@tok\expandafter{\the\CatchFBT@tok}}%
      \fi%
      \CatchFBT@IsAToken{#1}%
         {\global#1\expandafter{\the\CatchFBT@tok}}%
         {\xdef#1{\the\CatchFBT@tok}}%
      \ifx\CatchFBT@tok#1\else\global\CatchFBT@tok{}\fi%
   \endgroup%
}% \CatchFBT@Final
\makeatother

\newtheorem{thm}{Theorem}[section]
\newtheorem{prop}[thm]{Proposition}
\newtheorem{lem}[thm]{Lemma}
\newtheorem{cor}[thm]{Corollary}
\theoremstyle{definition}
\newtheorem{definition}[thm]{Definition}
\newtheorem{example}[thm]{Example}
\theoremstyle{remark}
\newtheorem{remark}[thm]{Remark}
\numberwithin{equation}{section}

\def\IModeColorName{MidnightBlue}
\def\OModeColorName{Maroon}

\newcommand\IMode[1]{{\color{\IModeColorName}{#1}}}
\newcommand\OMode[1]{{\color{\OModeColorName}{#1}}}

\newcommand\HypJ[2]{#1\ \ (#2)}
\newcommand\GenJ[2]{\vert_{\IMode{#1}}\; #2}

\newcommand\IsTac[1]{\IMode{#1}\ \textit{tactic}}

\newcommand\MkSet[1]{\left\{#1\right\}}
\newcommand\Member[2]{\IMode{#1}\in\IMode{#2}}
\newcommand\EqMember[3]{\IMode{#1}=\IMode{#2}\in\IMode{#3}}
\newcommand\Of[2]{\IMode{#1}: \IMode{#2}}

\newcommand\MkValence[3]{%
  \ifthenelse{\equal{#1}{}}{}{%
    \{#1\}%
  }%
  \ifthenelse{\equal{#2}{}}{}{%
    [#2]%
  }.\, #3%
}
\newcommand\MkBTm[3]{\{#1\}[#2].\, #3}
\newcommand\MkArity[2]{(#1)\, #2}
\newcommand\IsArity[1]{\IMode{#1}\ \textit{arity}}
\newcommand\IsValence[1]{\IMode{#1}\ \textit{valence}}
\newcommand\IsSort[1]{\IMode{#1}\ \textit{sort}}
\newcommand\Leq[2]{\IMode{#1}\leq\IMode{#2}}

\newcommand\ArrId[1]{1_{#1}}
\newcommand\Match[2]{\IMode{#1}\equiv\OMode{#2}}

\newcommand\Symbols{\mathbf{I}}
\newcommand\SymSets{\mathbb{F}_\Symbols}
\newcommand\FinSets{\mathbb{F}}
\newcommand\Sorts{\mathcal{S}}
\newcommand\Valences{\mathcal{V}}
\newcommand\Arities{\mathcal{A}}
\newcommand\Operators{\mathcal{O}}

\newcommand\Cats{\mathbf{Cat}}
\newcommand\Discrete[1]{#1_{\equiv}}
\newcommand\Arr[1]{\mathbf{{#1}^{\rightarrow}}}
\newcommand\Sets{\mathbf{Set}}
\newcommand\SCtx{\mathbf{SCtx}}
\newcommand\Ctx{\mathbf{Ctx}}
\newcommand\Seq{\mathbf{Seq}}
\newcommand\HypCat{\mathbf{H}}

\newcommand\OpCat[1]{#1^{\mathsf{op}}}

\newcommand\IsOperator[3]{\IMode{#1}\Vdash\IMode{#2}: \OMode{#3}}

\newcommand\IsMetaCtx[1]{\IMode{#1}\ \textit{mctx}}
\newcommand\IsVarCtx[1]{\IMode{#1}\ \textit{vctx}}
\newcommand\IsSymCtx[1]{\IMode{#1}\ \textit{sctx}}

\newcommand\Lookup[3]{\IMode{#1}\ni\IMode{#2}:\OMode{#3}}

\newcommand\Grothendieck[2]{\oint^{#1} #2#1}
\newcommand\Pair[2]{\left\langle#1, #2 \right\rangle}
\newcommand\Tuple[1]{\left\langle#1 \right\rangle}
\newcommand\FLift[2]{
  \Operators\langle#1%
  \ifthenelse{\equal{#2}{}}{}{, #2}\rangle%
}
\newcommand\PairTimes[2]{\langle#1\, {\scriptstyle{{\times}}}\, #2 \rangle}
\newcommand\SSingleton{\bullet}
\newcommand\Forget[1]{|\, #1\, |}

\newcommand\Comma[2]{#1 \downarrow#2}
\newcommand\Coslice[2]{#1\, /\, #2}

\newcommand\IsAbt[5]{%
  \IMode{#1}\triangleright%
  \IMode{#2}\parallel%
  \IMode{#3}\vdash%
  \IMode{#4}:\OMode{#5}%
}
\newcommand\IsWf[5]{%
  \IMode{#1}\triangleright%
  \IMode{#2}\parallel%
  \IMode{#3}\vdash%
  \IMode{#4}\in_{\mathbf{wf}}\IMode{#5}%
}

\newcommand\IsBTm[5]{%
  \IMode{#1}\triangleright%
  \IMode{#2}\parallel%
  \IMode{#3}\vdash%
  \IMode{#4}:\IMode{#5}%
}

\newcommand\MV[1]{\mathfrak{#1}}
\newcommand\MApp[3]{#1\{#2\}(#3)}
\newcommand\App[2]{#1(#2)}
\newcommand\Dom[1]{\left\vert#1\right\vert}
\newcommand\FV[1]{\mathbf{FV}\left(#1\right)}
\newcommand\FS[1]{\mathbf{FS}\left(#1\right)}
\newcommand\NotIn[2]{\IMode{#1}\notin\IMode{#2}}
\newcommand\SortExp{\mathsf{exp}}

\newcommand\Hom[3]{#1\left[#2,#3\right]}

\newcommand\Rename[2]{%
  #2\upharpoonright#1%
}

\newcommand\IsRenaming[3]{%
  \Rename{\IMode{#1}}{\IMode{#2}}%
  \leadsto\OMode{#3}%
}

\newcommand\Subst[3]{%
  \left[#1\, /\, #2\right]#3%
}

\newcommand\IsSubst[4]{%
  \Subst{\IMode{#1}}{\IMode{#2}}{\IMode{#3}}%
  \leadsto\OMode{#4}%
}

\newcommand\IsEq[2]{\IMode{#1} = \IMode{#2}}
\newcommand\IsApart[2]{\IMode{#1} \mathrel{\#} \IMode{#2}}

\newcommand\Define[2]{\IMode{#1}\triangleq\OMode{#2}}
\newcommand\ADefine[2]{\IMode{#1}&\triangleq\OMode{#2}}

\newcommand\SigFun[1]{\mathcal{F}_{#1}}

\newcommand\SemBrackets[1]{\left\llbracket#1\right\rrbracket}
\newcommand\Yoneda[1]{\boldsymbol{y} (#1)}

\begin{document}

\title{Type Refinements for the Working Class}
\date{}
\author{Jon Sterling and Darin Morrison}
\maketitle

There are two conflicting views which bedevil any discussion of the nature of
type theory. First, there is the notion of type theory as an extension or
generalization of universal algebra to support interdependency of sorts and
operations, possibly subject to an arbitrary equational theory
\cite{cartmell:1986, dybjer:1996}; we will call this \emph{formal type theory}.
Typing, in such a setting, is a mere matter of grammar and is nearly always
decidable. In hindsight, we may observe that this is the sort of type theory
which Martin-L\"of first proposed in 1972 \cite{martin-lof:1972}, even if we
will admit that this was not the intention at the time. A model for such a type
theory is usually given by interpreting the types or sorts as presheaves or
sheaves over contexts of hypotheses, and as such, a proof theoretic
interpretation of the hypothetical judgment is inevitable.

Secondly, there is the view of type theory as semi-formal theory of
constructions for the Brouwer-Heyting-Kolmogorov interpretation of
intuitionistic mathematical language, which we will call \emph{behavioral} or
\emph{semantic type theory}. The most widely known development of this program
is Martin-L\"of's 1979 ``extensional'' type theory \cite{martin-lof:1979,
martin-lof:1984}, but we must give priority to Dana Scott for inventing this
line of research in 1970 with his prophetic report, \emph{Constructive
Validity} \cite{scott:1970}. More recently, behavioral type theory has been
developed much further in the Nuprl family \cite{constable:1986} of proof
assistants, including MetaPRL \cite{hickey:2003} and JonPRL \cite{jonprl:2015}.

Martin-L\"of's key innovation was the commitment to pervasive functionality
(extensionality) as part of the \emph{definitions} of the judgments and the
types, in contrast to the state of affairs in formal type theory where
functionality is a metatheorem which may or may not hold depending on the
particular equational theory which has been imposed. Furthermore, models for
behavioral type theory interpret the types as partial equivalence relations on
only closed terms, and the meaning of the hypothetical judgment is defined
separately and uniformly in the logical relations style.

Our position is that these views of type theory are not in conflict, but rather
merely describe two distinct layers in a single, harmonious system. From this
perspective, formal type theories can do little more than negotiate matters of
grammar, and therefore may serve as a syntactic (linguistic) framework
for mathematical language, being responsible for the management of variable
binding and substitution. On the other hand, the meaning of mathematical
statements shall be specified \emph{behaviorally} in the semantic type
theory.

The types of the semantic theory can then be said to \emph{refine} the types of
the syntactic theory, both by placing restrictions on membership and by
coarsening equivalence. Thus far, all developments of behavioral type theory
have been built on a \emph{unityped} syntactic framework, and so the relation
to type refinements has been difficult to see. In this paper, we contribute a
full theory of behavioral refinements over multi-sorted abstract binding trees,
a simple formal type theory \cite{harper:2016, sterling-morrison:2015}; this
hybrid system allows the deployment of a Nuprl-style type theory over any
signature of sorts and operators.

\ifdraft{}{
  \bibliographystyle{abbrv}
  \bibliography{refs}
}


\end{document}
