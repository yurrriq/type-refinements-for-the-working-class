\documentclass[11pt]{article}

\usepackage{notation/modes}
\usepackage{notation/judgments}
\usepackage{notation/common}
\usepackage{notation/abts}
\usepackage{refinements}

\usepackage[usenames,dvipsnames,svgnames,table]{xcolor}
\usepackage{leftidx}
\usepackage{amsmath}
\usepackage{amssymb}
\usepackage{amsthm}
\usepackage{catchfilebetweentags}
\usepackage{geometry}
\usepackage{graphicx}
\usepackage{hyperref}
\usepackage{ifdraft}
\usepackage{ifthen}
\usepackage{mathtools}
\usepackage{proof}
\usepackage{scalerel}
\usepackage{setspace}
\usepackage{stmaryrd}
\usepackage{supertabular}
\usepackage{tikz-cd}
\usepackage[length=.5]{tikzpfeile}
\usepackage{url}
\usepackage[titletoc,title]{appendix}
\usepackage{todonotes}

%% fonts
\usepackage{pxfonts}
\usepackage{eulervm}
\usepackage[
  activate={true,nocompatibility},
  kerning=true,
  spacing=true,
  tracking=true
]{microtype}
\usepackage{bbold}

\newtheorem{thm}{Theorem}[section]
\newtheorem{prop}[thm]{Proposition}
\newtheorem{lem}[thm]{Lemma}
\newtheorem{cor}[thm]{Corollary}
\theoremstyle{definition}
\newtheorem{definition}[thm]{Definition}
\newtheorem{example}[thm]{Example}
\theoremstyle{notation}
\newtheorem{notation}[thm]{Notation}
\newtheorem*{notation*}{Notation}
\theoremstyle{remark}
\newtheorem{remark}[thm]{Remark}
\newtheorem*{remark*}{Remark}
\numberwithin{equation}{section}

\newcommand\HPGenJ[4]{\vert^{\IMode{#1}\parallel\IMode{#2}}_{\IMode{#3}}\; #4}

% power object
\newcommand\Pow[1]{\wp\left(#1\right)}

\newcommand\MapsTo[2]{\IMode{#1}\mapsto\OMode{#2}}
\newcommand\IsSubsetEq[2]{\IMode{#1}\subseteq\IMode{#2}}

\newcommand\CanOperators{\mathcal{K}}

\newcommand\IsAbtUnmoded[5]{
  #1\triangleright%
  #2\parallel%
  #3\vdash%
  #4:\OMode{#5}%
}

\newcommand\TyBase[1]{\mathtt{Base}_{#1}}
\newcommand\TyTop[1]{\top_{#1}}
\newcommand\TyBot[1]{\bot_{#1}}
\newcommand\TyMeet[1]{\cap_{#1}}
\newcommand\TyJoin[1]{\cup_{#1}}

\newcommand\EvalN[5]{\IMode{#1}\parallel\IMode{#4}\Downarrow^{\OMode{#3}}_{\IMode{#2}}\OMode{#5}}
\newcommand\Eval[4]{\EvalN{#1}{#2}{}{#3}{#4}}
\newcommand\Exprs{\mathbf{E}}
\newcommand\Values{\mathbf{V}}
\newcommand\BTms{\mathbf{B}}
\newcommand\Naturals{\mathbb{N}}
\newcommand\NaturalsPsh{\mathcal{N}}

\tikzset{
  commutative diagrams/.cd,
  arrow style = tikz,
  diagrams = {>=latex},
}
\tikzstyle{displays} = [
  ->,
  >=open triangle 45,
]

\begin{document}

\title{Type Refinements for the Working Class}
\date{}
\author{Jon Sterling and Darin Morrison}
\maketitle

There are two conflicting views which bedevil any discussion of the nature of
type theory. First, there is the notion of type theory as an extension or
generalization of universal algebra to support interdependency of sorts and
operations, possibly subject to an arbitrary equational theory
\cite{cartmell:1986, dybjer:1996}; we will call this \emph{formal type theory}.
Typing, in such a setting, is a mere matter of grammar and is nearly always
decidable. In hindsight, we may observe that this is the sort of type theory
which Martin-L\"of first proposed in 1972~\cite{martin-lof:1972}, even if we
will admit that this was not the intention at the time. A model for such a type
theory is usually given by interpreting the types or sorts as presheaves or
sheaves over contexts of hypotheses, and as such, a proof theoretic
interpretation of the hypothetical judgment is possible.

Secondly, there is the view of type theory as semi-formal theory of
constructions for the Brouwer-Heyting-Kolmogorov interpretation of
intuitionistic mathematical language, which we will call \emph{behavioral} or
\emph{semantic type theory}. The most widely known development of this program
is Martin-L\"of's 1979 ``extensional'' type theory~\cite{martin-lof:1979,
martin-lof:1984}, but we must give priority to Dana Scott for inventing this
line of research in 1970 with his prophetic report, \emph{Constructive
Validity}~\cite{scott:1970}. Since the 1980s, behavioral type theory has been
developed much further in the Nuprl family~\cite{constable:1986} of proof
assistants, including MetaPRL~\cite{hickey:2003} and JonPRL~\cite{jonprl:2015}.

Martin-L\"of's key innovation was the commitment to pervasive functionality
(extensionality) as part of the \emph{definitions} of the judgments and the
types, in contrast to the state of affairs in formal type theory where
functionality is a metatheorem which must be shown to obtain, based on the
(somewhat arbitrary) equational theory which has been imposed. Furthermore,
models for behavioral type theory interpret the types as partial equivalence
relations (PERs) on only closed terms, and the meaning of the hypothetical
judgment is defined separately and uniformly in the logical relations style.

Our position is that these views of type theory are not in conflict, but rather
describe two distinct layers in a single, harmonious system. From this
perspective, formal type theories can do little more than negotiate matters of
grammar, and therefore may serve as a linguistic framework for mathematical
expression, being responsible for the management of variable binding and
substitution, among other things. On the other hand, mathematical objects find
\emph{meaning} in their extension within the behavioral type theory, and are
subject to an extensional equality.

The types of the semantic theory can then be said to \emph{refine} the types of
the syntactic theory~\cite{harper-davies:2014, harper-duff:2015, harper:2016},
both by placing restrictions on membership and by coarsening equivalence. Thus
far, most developments of behavioral type theory have been built on a
\emph{unityped} syntactic framework, and so the relation to type refinements
has been difficult to see.

In this paper, we contribute a full theory of behavioral refinements over
multi-sorted nominal abstract binding trees with symbolic parameters, a simple
but interesting formal type theory~\cite{harper:2016, sterling-morrison:2015};
this hybrid system allows the deployment of a Nuprl-style type theory over any
signature of sorts and operators.

Owing to the \emph{nominal} aspect of our abstract binding tree (abt) framework, this
development constitutes a possible alternative to Allen's semantics for
unguessable atoms in Nuprl~\cite{allen:2006}; Allen proposed a
``supervaluation'' semantics, that explained sequents involving atoms (symbols)
by quantifying over the possible ways to interpret the class of atoms. Our
semantics seem to be a more direct and concrete way to approach the problem,
which is licensed in part by the nominal character of the abt framework.

\ifcolored%
\paragraph{Colors}
In this paper, we hint at the \emph{modes} of judgments and
assertions~\cite{harper:2016} using colors, marking inputs with
$\IMode{\texttt{blue}}$ and outputs with $\OMode{\texttt{red}}$. As a rule of
thumb, inputs are things which are supplied when checking the correctness of a
judgment, and outputs are things which are synthesized in the process. \fi

\section{Abstract Binding Trees and Symbols}

See~\cite{sterling-morrison:2015} for the development of abstract binding trees
with symbols. \todo[inline]{give a brief description of the framework, and present its
rules.}

\section{Behavioral Refinements}

Fixing a signature $\Match{\Sigma}{\Pair{\Sorts}{\Operators}}$ in the abt
framework, we will define the notion of behavioral refinement by propounding
several judgments and their semantical explanations. Let us first define the
copresheaf of $\tau$-sorted expressions, $\Exprs_\tau$ as follows
\[
  \Define{
    \Exprs_\tau(\Upsilon\parallel\Gamma)
  }{
    \MkSet{M \mid \IsAbtUnmoded{\cdot}{\Upsilon}{\Gamma}{M}{\tau}}
  }
\]

We will also write $\Exprs_\tau$ for the copresheaf $\Exprs_\tau(-\parallel\cdot)$ on $\SCtx$.

\subsection{Parametric Refinement}
\label{sec:parametric-refinement}

The first judgment that will concern us is called \emph{parametric refinement},
$\Refines{\Upsilon}{\phi}{\tau}$, which means that $\phi$ refines the sort
$\tau$ under the symbolic parameters $\Upsilon$. We define this judgment
through a meaning explanation in the style of Martin-L\"of as follows:

\newcommand\RefEquate[4]{#1\{\IMode{#2}\}\left(\IMode{#3},\IMode{#4}\right)}

\begin{definition}
  To know $\Refines{\Upsilon}{\phi}{\tau}$ (presupposing $\IsSymCtx{\Upsilon}$
  and $\IsSort{\tau}$) is to know, for any
  \begin{tikzcd}[cramped]
    \IMode{\Upsilon}\arrow[r,hook,"\IMode{\rho}"] &\IMode{\Upsilon'}
  \end{tikzcd}
, what it means for any $\Member{M,N}{\Exprs_\tau(\Upsilon')}$ to be equated by
  $\phi$ (written $\RefEquate{\phi}{\rho}{M}{N}$), such that
  $\RefEquate{\phi}{\rho}{-}{-}$ is a partial equivalence relation (PER) on
  $\Exprs_\tau(\Upsilon')$. Morever, that for any
  \begin{tikzcd}[cramped]
    \IMode{\Upsilon'}\arrow[r,hook,"\IMode{\rho'}"] &\IMode{\Upsilon''}
  \end{tikzcd}
,
  from $\RefEquate{\phi}{\rho}{M}{N}$ we may conclude
  $\RefEquate{\phi}{\rho'\circ\rho}{\Rename{\rho'}{M}}{\Rename{\rho'}{N}}$.
\end{definition}

Then, we say that $\phi$ \emph{globally refines} $\tau$ (written
$\GloballyRefines{\phi}{\tau}$) when we have $\Refines{\cdot}{\phi}{\tau}$.
Furthermore, when $\Refines{\Upsilon}{\phi}{\tau}$, for any renaming
\begin{tikzcd}[cramped]
  \IMode{\Upsilon}\arrow[r,hook,"\IMode{\rho}"] &\IMode{\Upsilon'}
\end{tikzcd}
we can clearly define a new refinement $\Refines{\Upsilon'}{\Rename{\rho}{\rho}}{\tau}$.

\begin{remark}\label{rem:internal}

  At this point, it should be noted that this is very similar to the semantical
  explanation of typehood given in~\cite{martin-lof:1979}, except that we have
  generalized it to a multi-sorted setting, and that we have fibred the entire
  apparatus over collections $\Upsilon$ of symbols; furthermore, whereas
  Martin-L\"of defines types in terms of evaluation to canonical form, we have
  remained agnostic on this point as far as refinements are concerned. We will
  come back to the notions of \emph{type} and \emph{computation} in
  section~\ref{sec:ctt}.

  In fact, the complexity of the above meaning explanation is an artifact of
  the pointwise style in which we have expressed it. Considered internally to
  the copresheaf topos $\Sets^\SCtx$, a refinement is merely a section of the
  object of $\Exprs_\tau$-PERs $\PERs{\Exprs_\tau}$, defined
  internally as a subobject of $\Omega^{\Exprs_\tau\times\Exprs_\tau}$:
  \[
    \Define{
      \PERs{X}
    }{
      \{ \phi : \Omega^{X\times X}
         \mid
         symmetric(\phi)
         \land transitive(\phi)
      \}
    }
  \]

  So, $\Refines{\Upsilon}{\phi}{\tau}$ obtains just when we have
  $\Member{\phi}{\PERs{\Exprs_\tau}(\Upsilon)}$. The benefit of explaining such
  judgments as objects in the topos is that we do not need to deal with
  the complexities of quantifying over renamings, since this is implicit in the
  definition of the exponential of presheaves, where
  $\Define{\Coyoneda{\Upsilon}}{\Upsilon\inj-}$ is the co-yoneda embedding:
  \begin{align*}
    \ADefine{
      B^A(\Upsilon)
    }{
      \Hom{\Sets^\SCtx}{
        \Coyoneda{\Upsilon}\times A
      }{
        B
      }
    }\\
    &\cong
    \OMode{
      \int_{\Upsilon'}
      (\Upsilon\inj\Upsilon')
      \implies
      {B(\Upsilon')}^{A(\Upsilon')}
    }
  \end{align*}

  It should be noted that PERs can be understood as 0-semigroupoids. In other
  words, these are 1-groupoids without identity morphisms and without coherence
  laws regarding composition and inverse operations. This distinction is
  especially relevant in formal type theory where groupoids ``one dimension
  down'' are not sets but setoids (i.e., groupoids sans coherence). Thus, a PER
  can formally be thought of as a kind of partial setoid that facilitates
  constructions on subsets in addition to quotients. \todo[inline]{Say something
    about internal (semi)-groupoids?}

\end{remark}

\subsubsection{Order and Equality of Parametric Refinements}

We will write $\IsSubrefinement{\Upsilon}{\phi}{\psi}{\tau}$ to mean that $\phi$ is
a \emph{subrefinement} of $\psi$.

\begin{definition}

  To know $\IsSubrefinement{\Upsilon}{\phi}{\psi}{\tau}$ (presupposing
  $\Refines{\Upsilon}{\phi}{\tau}$ and $\Refines{\Upsilon}{\psi}{\tau}$), is
  to know, for any renamings
  \begin{tikzcd}[cramped]
    \IMode{\Upsilon} \arrow[r, hook, "\IMode{\rho}"]
    & \IMode{\Upsilon'} \arrow[r, hook, "\IMode{\rho'}"]
    & \IMode{\Upsilon''}
  \end{tikzcd}
  , that from $\RefEquate{\phi}{\rho}{M}{N}$ you can conclude
  $\RefEquate{\psi}{\rho'\circ\rho}{\Rename{\rho'}{M}}{\Rename{\rho'}{N}}$ for any
  $\Member{M,N}{\Exprs_\tau(\Upsilon')}$.

\end{definition}

Two refinements are equal when they denote the same PER.\@That is, we have
$\EqRefines{\Upsilon}{\phi}{\psi}{\tau}$ just when both
$\IsSubrefinement{\Upsilon}{\phi}{\psi}{\tau}$ and
$\IsSubrefinement{\Upsilon}{\psi}{\phi}{\tau}$.

\subsection{Parametric Equality}

The primary judgment concerning refinements is the parametric equality,
$\RMemEq{\Upsilon}{M}{N}{\phi}$, which presupposes
$\Refines{\Upsilon'}{\phi}{\tau}$ for some $\Upsilon'$ such that we have
\begin{tikzcd}[cramped, sep = small]
  \IMode{\Upsilon'} \arrow[r, hook, "\IMode{\rho}"] &\IMode{\Upsilon}
\end{tikzcd}
, and $\Member{M,N}{\Exprs_\tau(\Upsilon)}$. The meaning of this judgment is
that $M$ and $N$ are identified by $\phi$ at $\Upsilon$:
\[
  \infer{
    \RMemEq{\Upsilon}{M}{N}{\phi}
  }{
    \RefEquate{\phi}{\rho}{M}{N}
  }
\]

\subsection{Functional Refinement}

\newcommand\RefinesCtxNil{\sqsubset^\star_{\mathtt{nil}}}
\newcommand\RefinesCtxSnoc{\sqsubset^\star_{\mathtt{snoc}}}
\newcommand\PFunRefNil{\vDash^\sqsubset_{\mathtt{nil}}}
\newcommand\PFunRefSnoc{\vDash^\sqsubset_{\mathtt{snoc}}}

Next, we define functional refinement
$\FEqRefines{\Upsilon}{\Psi}{\phi}{\psi}{\tau}$ in terms of parametric refinement,
simultaneously with parametric context refinement
$\RefinesCtx{\Upsilon}{\Psi}{\Gamma}$ and functional equality
$\FRMemEq{\Upsilon}{\Psi}{M}{N}{\phi}$. We will write
$\FRefines{\Upsilon}{\Psi}{\phi}{\tau}$ as a shorthand for
$\FEqRefines{\Upsilon}{\Psi}{\phi}{\phi}{\tau}$.

\newcommand\DefRefinesCtxNil[1]{
  \infer[\RefinesCtxNil]{
    \RefinesCtx{#1}{\cdot}{\cdot}
  }{
  }
}
\newcommand\DefRefinesCtxSnoc[6]{
  \infer[\RefinesCtxSnoc]{
    \RefinesCtx{#1}{#2,#3:#4}{#5,#3:#6}
  }{
    \RefinesCtx{#1}{#2}{#5} &
    \FRefines{#1}{#2}{#4}{#6}
  }
}

Parametric context refinement $\RefinesCtx{\Upsilon}{\Psi}{\Gamma}$ shall be
defined as follows, presupposing $\IsSymCtx{\Upsilon}$ and
$\IsVarCtx{\Gamma}$:
\begin{gather*}
  \DefRefinesCtxNil{\Upsilon}\qquad
  \DefRefinesCtxSnoc{\Upsilon}{\Psi}{x}{\phi}{\Gamma}{\tau}
\end{gather*}

For a context refinement $\RefinesCtx{\Upsilon}{\Psi}{\Gamma}$ and a renaming
\begin{tikzcd}[cramped]
  \IMode{\Upsilon} \arrow[r,hook,"\IMode{\rho}"] &\IMode{\Upsilon'}
\end{tikzcd}
, we can define a new context refinement
$\RefinesCtx{\Upsilon'}{\Rename{\rho}{\Psi}}{\Gamma}$ by applying $\rho$ pointwise at
each of the sort refinements in the context.

Functional refinement $\FEqRefines{\Upsilon}{\Psi}{\phi}{\psi}{\tau}$ presupposes
$\RefinesCtx{\Upsilon}{\Psi}{\Gamma}$ and $\IsSort{\tau}$, and is explained by
induction on the evidence for its first presupposition.

\paragraph{Case} $\DefRefinesCtxNil{\Upsilon}$.
\[
  \infer[\PFunRefNil]{
    \FEqRefines{\Upsilon}{\cdot}{\phi}{\psi}{\tau}
  }{
    \EqRefines{\Upsilon}{\phi}{\psi}{\tau}
  }
\]

\paragraph{Case} $\DefRefinesCtxSnoc{\Upsilon}{\Psi}{x}{\chi}{\Gamma}{\sigma}$.

\begin{quote}
  To know $\FEqRefines{\Upsilon}{\Psi,x:\chi}{\phi}{\psi}{\tau}$ is to know, for any renaming
  \begin{tikzcd}[cramped]
    \IMode{\Upsilon} \arrow[r,hook,"\IMode{\rho}"] &\IMode{\Upsilon'}
  \end{tikzcd}
  and closed terms $\Member{M_0,M_1}{\Exprs_\sigma(\Upsilon')}$, that from
  $\FRMemEq{\Upsilon'}{\Rename{\rho}{\Psi}}{M_0}{M_1}{\Rename{\rho}{\chi}}$, you can conclude
  $\FEqRefines{\Upsilon'}{\Rename{\rho}{\Psi}}{\Subst{M_0}{x}{\Rename{\rho}{\phi}}}{\Subst{M_1}{x}{\Rename{\rho}{\psi}}}{\tau}$.
  In other words:
  %
  \[
    \infer[\PFunRefSnoc]{
      \FEqRefines{\Upsilon}{\Psi,x:\chi}{\phi}{\psi}{\tau}
    }{
      \begin{array}{l}
        \forall
        \begin{tikzcd}[cramped, ampersand replacement = \&]
          \IMode{\Upsilon} \arrow[r,hook,"\IMode{\rho}"] \&\IMode{\Upsilon'}
        \end{tikzcd}
        .\ \forall \Member{M_0,M_1}{\Exprs_\sigma(\Upsilon')}.\\
        \left(\FRMemEq{\Upsilon'}{\Rename{\rho}{\Psi}}{M_0}{M_1}{\Rename{\rho}{\chi}}\right)\\
        \quad\implies
        \left(\FEqRefines{\Upsilon'}{\Rename{\rho}{\Psi}}{\Subst{M_0}{x}{\Rename{\rho}{\phi}}}{\Subst{M_1}{x}{\Rename{\rho}{\psi}}}{\tau}\right)
      \end{array}
    }
  \]
\end{quote}


\subsection{Functional Equality}

Functional equality $\FRMemEq{\Upsilon}{\Psi}{N_0}{N_1}{\phi}$ presupposes
$\RefinesCtx{\Upsilon}{\Psi}{\Gamma}$, $\FRefines{\Upsilon}{\Psi}{\phi}{\tau}$,
and $\Member{N_0,N_1}{\Exprs_\tau(\Upsilon\parallel\Gamma)}$; this judgment is defined by
induction on the evidence for the first presupposition
$\RefinesCtx{\Upsilon}{\Psi}{\Gamma}$:


\paragraph{Case} $\DefRefinesCtxNil{\Upsilon}$.
\[
  \infer{
    \FRMemEq{\Upsilon}{\cdot}{N_0}{N_1}{\phi}
  }{
    \RMemEq{\Upsilon}{N_0}{N_1}{\phi}
  }
\]

\paragraph{Case} $\DefRefinesCtxSnoc{\Upsilon}{\Psi}{x}{\chi}{\Gamma}{\sigma}$.
\begin{quote}
  To know $\FRMemEq{\Upsilon}{\Psi,x:\chi}{N_0}{N_1}{\phi}$ is to know, for any renaming
  \begin{tikzcd}[cramped]
    \IMode{\Upsilon} \arrow[r,hook,"\IMode{\rho}"] &\IMode{\Upsilon'}
  \end{tikzcd}
  and closed terms $\Member{M_0,M_1}{\Exprs_\sigma(\Upsilon')}$, that from
  $\FRMemEq{\Upsilon'}{\Rename{\rho}{\Psi}}{M_0}{M_1}{\Rename{\rho}{\chi}}$, you can conclude
  $\FRMemEq{\Upsilon'}{\Rename{\rho}{\Psi}}{\Subst{M_0}{x}{\Rename{\rho}{N_0}}}{\Subst{M_1}{x}{\Rename{\rho}{N_1}}}{\Subst{M_0}{x}{\Rename{\rho}{\phi}}}$.
  In other words:
  %
  \[
    \infer{
      \FRMemEq{\Upsilon}{\Psi,x:\psi}{N_0}{N_1}{\phi}
    }{
      \begin{array}{l}
        \forall
        \begin{tikzcd}[cramped, ampersand replacement = \&]
          \IMode{\Upsilon} \arrow[r,hook,"\IMode{\rho}"] \&\IMode{\Upsilon'}
        \end{tikzcd}
        .\ \forall \Member{M_0,M_1}{\Exprs_\sigma(\Upsilon')}.\\
        \left(\FRMemEq{\Upsilon'}{\Rename{\rho}{\Psi}}{M_0}{M_1}{\Rename{\rho}{\chi}}\right)\\
        \quad\implies
        \left(%
          \FRMemEq{\Upsilon'}{\Rename{\rho}{\Psi}}{%
            \Subst{M_0}{x}{\Rename{\rho}{N_0}}
          }{%
            \Subst{M_1}{x}{\Rename{\rho}{N_1}}%
          }{\Subst{M_0}{x}{\Rename{\rho}{\phi}}
          }%
        \right)
      \end{array}
    }
  \]
\end{quote}

\section{Constructions on Refinements}

Refinements can be arranged in a partial order via the
$\IsSubrefinement{\Upsilon}{\phi}{\psi}{\tau}$ judgment, and admit a lattice structure
at any sort $\tau$.

We have the bottom refinement $\GloballyRefines{\TyBot{\tau}}{\tau}$, which
simply contains no elements. Next, the top refinement
$\GloballyRefines{\TyTop{\tau}}{\tau}$ identifies all terms
$\Member{M,N}{\Exprs_\tau(\Upsilon)}$.  The join
$\Refines{\Upsilon}{\phi\TyJoin{\tau}\psi}{\tau}$ of two refinements
$\Refines{\Upsilon}{\phi}{\tau}$ and $\Refines{\Upsilon}{\psi}{\tau}$ is defined as
follows:

\begin{gather*}
  \infer{
    \RMemEq{\Upsilon}{M}{N}{\phi\TyJoin{\tau}\psi}
  }{
    \RMemEq{\Upsilon}{M}{N}{\phi}
  }\qquad
  \infer{
    \RMemEq{\Upsilon}{M}{N}{\phi\TyJoin{\tau}\psi}
  }{
    \RMemEq{\Upsilon}{M}{N}{\psi}
  }
\end{gather*}

Finally the meet $\Refines{\Upsilon}{\phi\TyMeet{\tau}\psi}{\tau}$ of two refinements
$\Refines{\Upsilon}{\phi}{\tau}$ and $\Refines{\Upsilon}{\psi}{\tau}$ is defined
dually:
\[
  \infer{
    \RMemEq{\Upsilon}{M}{N}{\phi\TyMeet{\tau}\psi}
  }{
    \RMemEq{\Upsilon}{M}{N}{\phi} &
    \RMemEq{\Upsilon}{M}{N}{\psi}
  }
\]

It is trivial to verify that these constructions on refinements give rise to a
lattice.

\section{Multi-Sorted Computational Type Theory}
\label{sec:ctt}

Martin-L\"of's type theory is a theory of \emph{value types}: to define a type,
you specify how to form its equal canonical inhabitants, or values. Then, the
inhabitants of the types are explained uniformly by extending this relation
over an operational semantics such that two terms are equal members of a
type just when they compute to equal values of that type.

More generally, one may wish to consider types which include computations in
addition to values; an example of such a type would be $\psi_\bot$ when $\psi$
is a type, which would include all the members of $\psi$ as well as a divergent
computation. Another example is the type $\TyTop{\tau}$ which identifies all
closed terms of sort $\tau$.

In order to ensure that such types are sensible, we must formulate a notion of
computational equivalence, and then say that a refinement is a \emph{type} just
when it respects computational equivalence; then, we are able to define types
as PERs on \emph{programs} in addition to values.


\subsection{Lazy Computation Systems}

In this section we generalize Howe's notion of lazy computation
system~\cite{howe:1989} to the multi-sorted, symbol-parameterized setting. An
\emph{lazy computation language} is an abt signature
$\Match{\Sigma}{\Pair{\Sorts}{\Operators}}$ along with a distinguished
arity-indexed family of copresheaves
$\HypJ{\Of{\CanOperators_a}{\Sets^\SCtx}}{\Member{a}{\Arities}}$ of
\emph{canonical} operators such that
$\IsSubsetEq{\CanOperators_a}{\Operators_a}$ for each arity $a$.

For a lazy computation language
$\Match{L}{\Tuple{\Sorts,\Operators,\CanOperators}}$, let us define
sort-indexed (resp.\ valence-indexed) copresheaves on $\SCtx$ of closed values
(resp\. bound terms) as follows:
\begin{align*}
  \ADefine{\Values_\tau(\Upsilon)}{
    \MkSet{
      M\equiv\App{\vartheta}{\vec{E}}
      \mid M\in\Exprs_\tau(\Upsilon)\land \exists a. \CanOperators\Pair{\Upsilon}{a}\ni\vartheta
    }
  }\\
  \ADefine{
    \BTms_{v}(\Upsilon)
  }{
    \MkSet{E \mid \IsAbtUnmoded{\cdot}{\Upsilon}{\cdot}{E}{v}}
  }
\end{align*}

Then, a \emph{lazy computation system} (lcs) is a lazy computation language
$\Match{L}{\Pair{\Sigma}{\CanOperators}}$ along with an $\SCtx$-indexed
$\equiv_\alpha$-functional evaluation relation
$\EvalN{\Upsilon}{\tau}{n}{M}{N}$ presupposing $\Member{n}{\Naturals}$,
$\Member{M}{\Exprs_\tau(\Upsilon)}$ and $\Member{N}{\Values_\tau(\Upsilon)}$,
expressing that $M$ evaluates to $N$ in $n$ steps.  We will write
$\Eval{\Upsilon}{\tau}{M}{N}$ to mean that there exists an $n$ such that
$\EvalN{\Upsilon}{\tau}{n}{M}{N}$.

\begin{remark}
  In any topos $\mathcal{E}$, for an object $X$ we can define the object of
  relations on $X$ as the exponential $\Define{\Pow{X}}{\Omega^X}$. Then, the
  data of such a relation is contained in a global section
  $\Of{R}{\Hom{\mathcal{E}}{\mathbb{1}}{\Pow{X}}}$.
\end{remark}

\newcommand\SBinRel[4]{\IMode{#2}\parallel\IMode{#3}\mathrel{#1}\IMode{#4}}

Fix a sort-indexed family of binary relations on closed expressions
$\Of{R_\tau}{\Hom{\Sets^{\SCtx}}{\mathbb{1}}{\Pow{\Exprs_\tau^2}}}$; we will
also write the relation as a judgment scheme
$\SBinRel{R_\tau}{\Upsilon}{M}{N}$. Now, we can always extend $R_\tau$ to a new
family of relations
$\Of{R_v}{\Hom{\Sets^{\SCtx}}{\mathbb{1}}{\Pow{\BTms_v^2}}}$ for any valence
$\Match{v}{\MkValence{\vec{\sigma}}{\vec{\tau}}{\tau}}$, defined pointwise as
follows:
\[
  \infer{
    \SBinRel{R_{\MkValence{\vec{\sigma}}{\vec{\tau}}{\tau}}}{\Upsilon}{
      \MkBTm{\vec{u}}{\vec{x}}{M}
    }{
      \MkBTm{\vec{v}}{\vec{y}}{N}
    }
  }{
    \begin{array}{l}
      \forall\IMode{\vec{w}}\mathrel{\#}\IMode{\Dom{\Upsilon}\cup\vec{u}\cup\vec{v}}.\quad
      \forall
        \begin{tikzcd}[cramped, ampersand replacement = \&]
           \IMode{\Upsilon,\vec{w}:\vec{\sigma}} \arrow[r,hook,"\IMode{\rho}"] \&\IMode{\Upsilon'}
        \end{tikzcd}.\quad
      \forall\Of{\vec{X}}{\Exprs^{[\vec{\tau}]}(\Upsilon')}.\\
      \quad
      \SBinRel{R_\tau}{\Upsilon'}{
        \Subst{\vec{X}}{\vec{x}}{
          \Rename{(\rho\circ \vec{u}\mapsto\vec{w})}{M}
        }
      }{
        \Subst{\vec{X}}{\vec{y}}{
          \Rename{(\rho\circ \vec{u}\mapsto\vec{w})}{M}
        }
      }
    \end{array}
  }
\]

\begin{notation*}
  For a family of copresheaves $\HypJ{\Of{X_\tau}{\Sets^\SCtx}}{\Member{\tau}{\Sorts}}$, we will
  use $X^{[\vec{\tau}]}$ as a shorthand for the following iterated product, as
  in \cite{sterling-morrison:2015}:

  \[
    \Define{
      X^{[\vec{\tau}]}
    }{
      \prod_{\tau\in\vec{\tau}}
        X_\tau
    }
  \]
\end{notation*}

As a matter of convenience, we'll also define this judgment over vectors of
bound terms and valences:
\[
  \infer{
    \SBinRel{R_{\vec{v}}}{\Upsilon}{\vec{E}}{\vec{F}}
  }{
    \forall\Member{(E,F,v)}{(\vec{E},\vec{F},\vec{v})}.\
    \SBinRel{R_v}{\Upsilon}{E}{F}
  }
\]

Now, we can extend $R_\tau$ to a new relation $[R_\tau]$ on closed expressions
which respects a single ``layer'' of computation. We will say
$\SBinRel{[R_\tau]}{\Upsilon}{M}{N}$ when, supposing
$\Eval{\Upsilon}{\tau}{M}{\App{\vartheta}{\Vec{E}}}$ such that
$\IMode{\Upsilon}\Vdash_{\IMode{\CanOperators}}\IMode{\vartheta}:\OMode{\MkArity{\vec{v}}{\tau}}$,
for any
\begin{tikzcd}[cramped]
  \IMode{\Upsilon} \arrow[r,hook,"\IMode{\rho}"] &\IMode{\Upsilon'}
\end{tikzcd}
, we have $\Eval{\Upsilon'}{\tau}{\Rename{\rho}{N}}{\App{\Rename{\rho}{\vartheta}}{\Rename{\rho}{\Vec{F}}}}$ and
$\SBinRel{R_{\vec{v}}}{\Upsilon'}{\Rename{\rho}{\vec{E}}}{\Rename{\rho}{\vec{F}}}$.

\begin{definition}[Computational Approximation]
Because $[-_\tau]$ is monotonic, it has a greatest fixed point, which we will
call \emph{computational approximation},
$\Of{\preccurlyeq_\tau}{\Hom{\Sets^\SCtx}{\mathbb{1}}{\Pow{\Exprs_\tau^2}}}$,
written pointwise as $\SBinRel{\preccurlyeq_\tau}{\Upsilon}{M}{N}$.
\end{definition}

\begin{definition}[Computational Equivalence]
  Two terms are said to be \emph{computationally equivalent} when they approximate each other:
  \[
    \infer{
      \SBinRel{\sim_\tau}{\Upsilon}{M}{N}
    }{
      \SBinRel{\preccurlyeq_\tau}{\Upsilon}{M}{N} &
      \SBinRel{\preccurlyeq_\tau}{\Upsilon}{N}{M}
    }
  \]
\end{definition}

\subsection{When is a refinement a type?}

As we remarked in section~\ref{sec:parametric-refinement}, the notion of
\emph{refinement} that we described is very much akin to a multi-sorted
generalization of Martin-L\"of's explanation of types
in~\cite{martin-lof:1979}, except that we have not required an operational
semantics to be present prior to the definition of refinements.

Following Martin-L\"of, a type is a refinement which behaves well with respect
to computation, in some specified sense; contrary to Martin-L\"of, however, we wish to
consider a full range of \emph{computational types}, including those which are
not necessarily defined by their values; a general apparatus for
computation-respecting refinements will allow us to define, for instance, a
type for partial functions.

\begin{definition}[Typehood]
  For any refinement $\Refines{\Upsilon}{\phi}{\tau}$, we say
  $\IsType{\Upsilon}{\phi}{\tau}$ just when for any
  \begin{tikzcd}[cramped]
    \IMode{\Upsilon}\arrow[r,hook,"\IMode{\rho}"] &\IMode{\Upsilon'}
  \end{tikzcd}%
  , and for all $\Member{L,M,N}{\Exprs_\tau(\Upsilon')}$, from
  $\SBinRel{\sim_\tau}{\Upsilon'}{L}{M}$ and $\RMemEq{\Upsilon'}{M}{N}{\phi}$ we
  can conclude $\RMemEq{\Upsilon'}{L}{N}{\phi}$.
\end{definition}

Stated internally as an object in the topos, the meaning of parametric typehood for
$\phi$ is perhaps a bit more clear:
\[
  \Define{
    \IsType{-}{\phi}{\tau}
  }{
    \prod_{L,M,N\in\Exprs_\tau}
    {\phi(L,N)}^{L\sim_\tau M \times \phi(M,N)}
  }
\]

Let us write $\Types{\tau}$ for the object of types,
\[
  \Define{
    \Types{\tau}
  }{
    \MkSet{
      \phi\in\PERs{\Exprs_\tau}
      \mid
      -\parallel\phi\ \textit{type}_\tau
    }
  }
\]

Now, let us return briefly to Martin-L\"of's notion of type, which we call a
\emph{value type}. The minimal data of such a type's definition consists in a
partial equivalence relation on values,
$\Of{\phi}{\Hom{\Sets^\SCtx}{\Coyoneda{\Upsilon}}{\PERs{\Values_\tau}}}$.
It is easy to turn such a relation into a computational type
$\IsType{\Upsilon}{\VToCType(\phi)}{\tau}$, as follows:
\[
  \begin{tikzcd}[sep=huge]
    \IMode{\Coyoneda{\Upsilon}\times\Values_\tau^2}
      \arrow[r, "\IMode{\lambda(\phi)}"]
      \arrow[d, hook, swap, "\IMode{\Pair{\ArrId{}}{i}}"]
      &
    \IMode{\Omega}\\
    \IMode{\Coyoneda{\Upsilon}\times\Exprs_\tau^2}
    \arrow[ur, swap, dotted, "\OMode{\lambda(\VToCType(\phi))}"]
  \end{tikzcd}
\]
where $\VToCType(\phi)$ is the computational extension of $\phi$ in the following sense:
\begin{align*}
  \ADefine{
    \VToCType(\phi)(M,N)
  }{
    \forall M', N' \in \Values_\tau.\
    M\sim_\tau M' \land N\sim_\tau N' \implies \phi(M',N')
  }
\end{align*}

\begin{thm}
When viewed as an internal functor
$\PERs{\Values_\tau}\to\Types{\tau}$, $\VToCType$ is the left adjoint to
the canonical inclusion
$\Of{\iota}{\Types{\tau}\inj\PERs{\Values_\tau}}$.
\end{thm}
\begin{proof}
  We will exhibit a counit $\Of{\epsilon}{\VToCType\circ\iota\to
  \ArrId{\PERs{\Values_\tau}}}$ and a unit
  $\Of{\eta}{\ArrId{\Types{\tau}}\to\iota\circ\VToCType}$ for
  the adjunction $\IMode{\VToCType}\dashv\IMode{\iota}$.

  The counit witnesses the fact that for any refinement
  $\Of{\phi}{\Types{\tau}}$ and terms $\Member{M,N}{\Exprs_\tau}$, if
  $\VToCType(\iota(\phi))(M,N)$, then $\phi(M,N)$; this is evidently the case,
  because the premise only obtains if $M,N$ are computationally equal to
  values, and $\phi$ respects computational equivalence by virtue of being a
  type.  The unit witnesses the fact that for any value PER
  $\Of{\psi}{\PERs{\Values_\tau}}$ and values $\Member{M,N}{\Values_\tau}$, if
  $\psi(M,N)$, then $\iota(\VToCType(\psi))(M,N)$; this is clearly true, since
  $\VToCType(\psi)$ can do nothing but coarsen $\psi$'s relation on values.

\end{proof}

\begin{definition}[Value Closure]
  For any computational type $\IsType{\Upsilon}{\phi}{\tau}$, let
  $\Define{\underline{\phi}}{\VToCType(\iota(\phi))}$ be called $\phi$'s
  \emph{value closure}, which we will use shortly in order to give a formal definition
  to the notion of value types.
\end{definition}

\begin{definition}[Value Types]
  We can give a simple characterization of Martin-L\"of's \emph{value types} in
  terms of our more general computational types. For any type $\IsType{\Upsilon}{\phi}{\tau}$,
  we will say $\IsValueType{\Upsilon}{\phi}{\tau}$ in case $\phi$ is equal to
  its value closure, i.e.\ $\EqRefines{\Upsilon}{\phi}{\underline{\phi}}{\tau}$.
\end{definition}

\ifdraft{}{
  \newpage
  \nocite{maclane:1971}
  \bibliographystyle{abbrv}
  \bibliography{refs}
}


\end{document}
