\documentclass[11pt]{article}

\usepackage[usenames,dvipsnames,svgnames,table]{xcolor}
\usepackage{leftidx}
\usepackage{amsmath}
\usepackage{amssymb}
\usepackage{amsthm}
\usepackage{catchfilebetweentags}
\usepackage{geometry}
\usepackage{graphicx}
\usepackage{hyperref}
\usepackage{ifdraft}
\usepackage{ifthen}
\usepackage{mathtools}
\usepackage{proof}
\usepackage{scalerel}
\usepackage{setspace}
\usepackage{stmaryrd}
\usepackage{supertabular}
\usepackage{tikz-cd}
\usepackage{url}
\usepackage{todonotes}

\newif\ifcolored%
\coloredtrue%

%% fonts
\usepackage{pxfonts}
\usepackage{eulervm}
\usepackage[
  activate={true,nocompatibility},
  kerning=true,
  spacing=true,
  tracking=true
]{microtype}
\DeclareSymbolFont{bbold}{U}{bbold}{m}{n}
\DeclareSymbolFontAlphabet{\mathbbold}{bbold}

\newtheorem{thm}{Theorem}[section]
\newtheorem{prop}[thm]{Proposition}
\newtheorem{lem}[thm]{Lemma}
\newtheorem{cor}[thm]{Corollary}
\theoremstyle{definition}
\newtheorem{definition}[thm]{Definition}
\newtheorem{example}[thm]{Example}
\theoremstyle{remark}
\newtheorem{remark}[thm]{Remark}
\newtheorem*{remark*}{Remark}
\numberwithin{equation}{section}

\ifcolored%
  \def\IModeColorName{MidnightBlue}
  \def\OModeColorName{Maroon}
\else
  \def\IModeColorName{Black}
  \def\OModeColorName{Black}
\fi

\newcommand\IMode[1]{{\color{\IModeColorName}{#1}}}
\newcommand\OMode[1]{{\color{\OModeColorName}{#1}}}

\newcommand\HypJ[2]{#1\ \ (#2)}
\newcommand\GenJ[2]{\vert_{\IMode{#1}}\; #2}
\newcommand\HPGenJ[4]{\vert^{\IMode{#1}\parallel\IMode{#2}}_{\IMode{#3}}\; #4}

\newcommand\IsTac[1]{\IMode{#1}\ \textit{tactic}}

\newcommand\MkSet[1]{\left\{#1\right\}}
\newcommand\Member[2]{\IMode{#1}\in\IMode{#2}}
\newcommand\EqMember[3]{\IMode{#1}=\IMode{#2}\in\IMode{#3}}
\newcommand\Of[2]{\IMode{#1}: \IMode{#2}}

\newcommand\MkValence[3]{%
  \ifthenelse{\equal{#1}{}}{}{%
    \{#1\}%
  }%
  \ifthenelse{\equal{#2}{}}{}{%
    [#2]%
  }.\, #3%
}
\newcommand\MkBTm[3]{\{#1\}[#2].\, #3}
\newcommand\MkArity[2]{(#1)\, #2}
\newcommand\IsArity[1]{\IMode{#1}\ \textit{arity}}
\newcommand\IsValence[1]{\IMode{#1}\ \textit{valence}}
\newcommand\IsSort[1]{\IMode{#1}\ \textit{sort}}
\newcommand\Leq[2]{\IMode{#1}\leq\IMode{#2}}

\newcommand\ArrId[1]{1_{#1}}
\newcommand\Match[2]{\IMode{#1}\equiv\OMode{#2}}

\newcommand\Symbols{\mathbf{I}}
\newcommand\SymSets{\mathbb{F}_\Symbols}
\newcommand\FinSets{\mathbb{F}}
\newcommand\Sorts{\mathcal{S}}
\newcommand\Valences{\mathcal{V}}
\newcommand\Arities{\mathcal{A}}
\newcommand\Operators{\mathcal{O}}

\newcommand\Cats{\mathbf{Cat}}
\newcommand\Discrete[1]{#1_{\equiv}}
\newcommand\Arr[1]{\mathbf{{#1}^{\rightarrow}}}
\newcommand\Sets{\mathbf{Set}}
\newcommand\SCtx{\mathbf{SCtx}}
\newcommand\Ctx{\mathbf{Ctx}}
\newcommand\Seq{\mathbf{Seq}}
\newcommand\HypCat{\mathbf{H}}

\newcommand\OpCat[1]{#1^{\mathsf{op}}}

\newcommand\IsOperator[3]{\IMode{#1}\Vdash\IMode{#2}: \OMode{#3}}

\newcommand\IsMetaCtx[1]{\IMode{#1}\ \textit{mctx}}
\newcommand\IsVarCtx[1]{\IMode{#1}\ \textit{vctx}}
\newcommand\IsSymCtx[1]{\IMode{#1}\ \textit{sctx}}

\newcommand\Lookup[3]{\IMode{#1}\ni\IMode{#2}:\OMode{#3}}

\newcommand\Grothendieck[2]{\oint^{#1} #2#1}
\newcommand\Pair[2]{\left\langle#1, #2 \right\rangle}
\newcommand\Tuple[1]{\left\langle#1 \right\rangle}
\newcommand\FLift[2]{
  \Operators\langle#1%
  \ifthenelse{\equal{#2}{}}{}{, #2}\rangle%
}
\newcommand\PairTimes[2]{\langle#1\, {\scriptstyle{{\times}}}\, #2 \rangle}
\newcommand\SSingleton{\bullet}
\newcommand\Forget[1]{|\, #1\, |}

\newcommand\Comma[2]{#1 \downarrow#2}
\newcommand\Coslice[2]{#1\, /\, #2}

\newcommand\IsAbtUnmoded[5]{
  #1\triangleright%
  #2\parallel%
  #3\vdash%
  #4:\OMode{#5}%
}
\newcommand\IsAbt[5]{\IsAbtUnmoded{\IMode{#1}}{\IMode{#2}}{\IMode{#3}}{\IMode{#4}}{\OMode{#5}}}
\newcommand\IsBTm[5]{\IsAbtUnmoded{\IMode{#1}}{\IMode{#2}}{\IMode{#3}}{\IMode{#4}}{\IMode{#5}}}

\newcommand\MV[1]{\mathfrak{#1}}
\newcommand\MApp[3]{%
  #1
  \ifthenelse{\equal{#2}{}}{}{%
    \{#2\}%
  }%
  \ifthenelse{\equal{#3}{}}{}{%
    [#3]%
  }
}
\newcommand\App[2]{#1(#2)}
\newcommand\Dom[1]{\left\vert#1\right\vert}
\newcommand\FV[1]{\mathbf{FV}\left(#1\right)}
\newcommand\FS[1]{\mathbf{FS}\left(#1\right)}
\newcommand\NotIn[2]{\IMode{#1}\notin\IMode{#2}}
\newcommand\SortExp{\mathsf{exp}}

\newcommand\Hom[3]{#1\!\left[#2,#3\right]}

\newcommand\Rename[2]{%
  #2\upharpoonright#1%
}

\newcommand\IsRenaming[3]{%
  \Rename{\IMode{#1}}{\IMode{#2}}%
  \leadsto\OMode{#3}%
}

\newcommand\Subst[3]{%
  \left[#1\, /\, #2\right]#3%
}

\newcommand\IsSubst[4]{%
  \Subst{\IMode{#1}}{\IMode{#2}}{\IMode{#3}}%
  \leadsto\OMode{#4}%
}

\newcommand\IsEq[2]{\IMode{#1} = \IMode{#2}}
\newcommand\IsApart[2]{\IMode{#1} \mathrel{\#} \IMode{#2}}

\newcommand\Define[2]{\IMode{#1}\triangleq\OMode{#2}}
\newcommand\ADefine[2]{\IMode{#1}&\triangleq\OMode{#2}}

\newcommand\SigFun[1]{\mathcal{F}_{#1}}

\newcommand\SemBrackets[1]{\left\llbracket#1\right\rrbracket}
\newcommand\Yoneda[1]{\boldsymbol{y} (#1)}

\newcommand\Refines[3]{\IMode{#1}\parallel\IMode{#2}\sqsubset\OMode{#3}}
\newcommand\GRefines[2]{\IMode{#1}\sqsubset\OMode{#2}}
\newcommand\RefinesCtx[2]{\IMode{#1}\sqsubset^\star\OMode{#2}}
\newcommand\EqRefines[4]{\IMode{#1}\parallel\IMode{#2}=\IMode{#3}\sqsubset\OMode{#4}}

\newcommand\HRefines[4]{\IMode{#1}\parallel\IMode{#2}\vDash\IMode{#3}\sqsubset\OMode{#4}}
\newcommand\HEqRefines[5]{\IMode{#1}\parallel\IMode{#2}\vDash\IMode{#3}=\IMode{#4}\sqsubset\OMode{#5}}
\newcommand\PRefinesCtx[3]{\IMode{#1}\parallel\IMode{#2}\sqsubset^\star\OMode{#3}}

\newcommand\RMemEq[4]{\IMode{#1}\parallel\IMode{#2}=\IMode{#3}\in\IMode{#4}}
\newcommand\HRMemEq[5]{\IMode{#1}\parallel\IMode{#2}\vDash\IMode{#3}=\IMode{#4}\in\IMode{#5}}

\newcommand\Pow[1]{\wp\left(#1\right)}

\begin{document}

\title{Type Refinements for the Working Class}
\date{}
\author{Jon Sterling and Darin Morrison}
\maketitle

There are two conflicting views which bedevil any discussion of the nature of
type theory. First, there is the notion of type theory as an extension or
generalization of universal algebra to support interdependency of sorts and
operations, possibly subject to an arbitrary equational theory
\cite{cartmell:1986, dybjer:1996}; we will call this \emph{formal type theory}.
Typing, in such a setting, is a mere matter of grammar and is nearly always
decidable. In hindsight, we may observe that this is the sort of type theory
which Martin-L\"of first proposed in 1972~\cite{martin-lof:1972}, even if we
will admit that this was not the intention at the time. A model for such a type
theory is usually given by interpreting the types or sorts as presheaves or
sheaves over contexts of hypotheses, and as such, a proof theoretic
interpretation of the hypothetical judgment is inevitable.

Secondly, there is the view of type theory as semi-formal theory of
constructions for the Brouwer-Heyting-Kolmogorov interpretation of
intuitionistic mathematical language, which we will call \emph{behavioral} or
\emph{semantic type theory}. The most widely known development of this program
is Martin-L\"of's 1979 ``extensional'' type theory~\cite{martin-lof:1979,
martin-lof:1984}, but we must give priority to Dana Scott for inventing this
line of research in 1970 with his prophetic report, \emph{Constructive
Validity}~\cite{scott:1970}. More recently, behavioral type theory has been
developed much further in the Nuprl family~\cite{constable:1986} of proof
assistants, including MetaPRL~\cite{hickey:2003} and JonPRL~\cite{jonprl:2015}.

Martin-L\"of's key innovation was the commitment to pervasive functionality
(extensionality) as part of the \emph{definitions} of the judgments and the
types, in contrast to the state of affairs in formal type theory where
functionality is a metatheorem which may or may not hold depending on the
particular equational theory which has been imposed. Furthermore, models for
behavioral type theory interpret the types as partial equivalence relations on
only closed terms, and the meaning of the hypothetical judgment is defined
separately and uniformly in the logical relations style.

Our position is that these views of type theory are not in conflict, but rather
merely describe two distinct layers in a single, harmonious system. From this
perspective, formal type theories can do little more than negotiate matters of
grammar, and therefore may serve as a syntactic (linguistic) framework
for mathematical language, being responsible for the management of variable
binding and substitution. On the other hand, the meaning of mathematical
statements shall be specified \emph{behaviorally} in the semantic type
theory.

The types of the semantic theory can then be said to \emph{refine} the types of
the syntactic theory~\cite{harper-davies:2014, harper-duff:2015, harper:2016},
both by placing restrictions on membership and by coarsening equivalence. Thus
far, all developments of behavioral type theory have been built on a
\emph{unityped} syntactic framework, and so the relation to type refinements
has been difficult to see. In this paper, we contribute a full theory of
behavioral refinements over multi-sorted abstract binding trees, a simple
formal type theory~\cite{harper:2016, sterling-morrison:2015}; this hybrid
system allows the deployment of a Nuprl-style type theory over any signature of
sorts and operators.

\section{Abstract Binding Trees and Symbols}

See~\cite{sterling-morrison:2015} for the development of abstract binding trees
with symbols. \todo[inline]{give a brief description of the framework, and present its
rules.}

\section{The Ambient Logical Framework}

\ifcolored%
In this paper, we hint at the \emph{modes} of judgments and
assertions~\cite{harper:2016} using colors, marking inputs with
$\IMode{\texttt{blue}}$ and outputs with $\OMode{\texttt{red}}$. As a rule of
thumb, inputs are things which are supplied when checking the correctness of a
judgment, and outputs are things which are synthesized in the process. \fi

Then, for any abt signature $\Match{\Sigma}{\Pair{\Sorts}{\Operators}}$, we can
deploy a generic logical framework. Recall from \cite{sterling-morrison:2015}
that $\SCtx$ is the category of symbol-contexts and their injective,
sort-preserving renamings. Then, fix a small category $\mathcal{W}$ of worlds
along with a map $\Of{\pi_\mathcal{W}}{\mathcal{W}\to\SCtx}$ such that
$\pi_\mathcal{W}$ is a Grothendieck opfibration. Then, a \emph{judgment} shall
be interpreted as an object in the presheaf category
$\Define{\mathbb{J}}{\Sets^\mathcal{W}}$; in other words, a judgment is
identified with the intensional set of its renaming-sensitive derivations.

\begin{definition}[General Judgment]
  For a metavariable context $\IsMetaCtx{\Omega}$, we can define the set
  of its substitutions as follows:
  \[
    \Define{{\square\Omega}}{
      \prod_{(\MV{m}:v)\in\Omega}
        \MkSet{E\mid\IsAbtUnmoded{\cdot}{\Upsilon}{\cdot}{E}{v}}
    }
  \]

  For any diagram $\Of{\mathcal{J}}{\mathbb{J}^{\square\Omega}}$, we can define
  the general judgment $\GenJ{\Omega}{\mathcal{J}}$ as follows, where
  $\Of{\Yoneda{-}}{\mathcal{W}\to\mathbb{J}}$ is the Yoneda Embedding:
   \[
     \textstyle
     \Define{
       \left(\GenJ{\Omega}{\mathcal{J}}\right)(w)
     }{
       \Hom{\mathbb{J}}{
         \Yoneda{w}
       }{
         Lan_{\mathbbold{1}}{\mathcal{J}}
       }
     }
   \]

   Intuitively, the general judgment $\GenJ{\Omega}{\mathcal{J}}$ is evident
   when the family of judgments $\mathcal{J}$ shall forevermore have a section
   (i.e. for any substitution $\Member{\omega}{\square\Omega}$,
   $\mathcal{J}(\omega)$ shall be evident). Recall that metavariables have
   \emph{valences} rather than sorts; as such, the general judgment allows us to
   quantify over terms of higher type.

   \paragraph{Notation}

   We will write $\HPGenJ{\Upsilon}{\Gamma}{\Omega}{\mathcal{J}}$ for
   $\GenJ{\Omega'}{\mathcal{J}}$, where
   $\Lookup{\Omega'}{\MV{m}}{\MkValence{\Upsilon,\vec{\sigma}}{\Gamma,\vec{\tau}}{\tau}}$
   if $\Lookup{\Omega}{\MV{m}}{\MkValence{\vec{\sigma}}{\vec{\tau}}{\tau}}$.
   Additionally, for clarity we will often write
   $\GenJ{\MV{m}:\sigma}{\mathcal{J}}$ rather than
   $\GenJ{\MV{m}:\MkValence{\cdot}{\cdot}{\sigma}}{\mathcal{J}}$; likewise, in
   the body of $\mathcal{J}$, when there are contextually salient
   $\vec{u}$, $\vec{x}$, we will simply write $\MV{m}$ rather than the more proper
   $\MApp{\MV{m}}{\vec{u}}{\vec{x}}$.

\end{definition}

\begin{definition}[Hypothetical Judgment]

  For two judgments $\mathcal{J}_1$ and $\mathcal{J}_2$, we can form the
  hypothetical judgment $\HypJ{\mathcal{J}_2}{\mathcal{J}_1}$, which expresses
  the semantic consequence of $\mathcal{J}_2$ from $\mathcal{J}_1$:
  \begin{align*}
    \ADefine{
      \left(\HypJ{\mathcal{J}_2}{\mathcal{J}_1}\right)(w)
    }{
      \left({\mathcal{J}_2}^{\mathcal{J}_1}\right)(w)
    }\\
    &\cong
    \OMode{
      \Hom{\mathbb{J}}{
        \Yoneda{w}\times\mathcal{J}_1
      }{
        \mathcal{J}_2
      }
    }
  \end{align*}

  In other words, the hypothetical judgment
  $\HypJ{\mathcal{J}_2}{\mathcal{J}_1}$ is evident when there shall forevermore
  be an effective transformation of evidence for the antecedent into evidence
  for the consequent. It is important to keep in mind that this is a
  \emph{semantic} consequence and therefore expresses admissibilities, in
  contrast to the \emph{logical} consequence which expresses
  derivabilities~\cite{harper:2016}. We will not have any need for the latter.
\end{definition}

After this section, we will define a judgment either by rules, or by an
informal semantical explanation, in both cases leaving its intensional
character implicit. A judgment is said to be \emph{correct} or \emph{valid}
just in case it is globally inhabited.

\section{Behavioral Refinements}

Fixing a signature $\Match{\Sigma}{\Pair{\Sorts}{\Operators}}$ in the abt
framework, we will define the notion of behavioral refinement by propounding
several judgments and their semantical explanations.

\subsection{Parametric Refinement}

We will require two judgments, defined mutually: $\RefinesCtx{\Phi}{\Upsilon}$,
which means that $\Phi$ is a symbol context which refines the symbol context
$\Upsilon$ (presupposing $\IsSymCtx{\Upsilon}$), and parametric refinement
$\Refines{\Phi}{\phi}{\tau}$ (which shall presuppose
$\RefinesCtx{\Phi}{\Upsilon}$ and $\IsSort{\tau}$).

\newcommand\RefinesCtxNil{\sqsubset^\star_{\mathtt{nil}}}
\newcommand\RefinesCtxSnoc{\sqsubset^\star_{\mathtt{snoc}}}

The first judgment we define inductively:
\begin{gather*}
  \infer[\RefinesCtxNil]{
    \RefinesCtx{\cdot}{\cdot}
  }{
  }\qquad
  \infer[\RefinesCtxSnoc]{
    \RefinesCtx{\Phi,u:\phi}{\Upsilon,u:\tau}
  }{
    \RefinesCtx{\Phi}{\Upsilon} &
    \Refines{\Phi}{\phi}{\tau}
  }
\end{gather*}

The second judgment is defined coinductively via the following meaning
explanation:
\begin{definition}[Parametric Refinement]
  To know $\Refines{\Phi}{\phi}{\tau}$ (presupposing
  $\RefinesCtx{\Phi}{\Upsilon}$ and $\IsSort{\tau}$) is to know, for any renaming
  $\Of{\varrho}{\Upsilon\hookrightarrow\OMode{\Upsilon'}}$, and for any $M, N$,
  what it means for $M$ and $N$ to be identified under $\phi$, supposing
  $\IsAbt{\cdot}{\Upsilon'}{\cdot}{M}{\tau}$ and
  $\IsAbt{\cdot}{\Upsilon'}{\cdot}{N}{\tau}$, requiring that if
  $\IMode{M}\sim^\rho_\phi\IMode{N}$ then $\IMode{N}\sim^\rho_\phi\IMode{M}$, and if
  $\IMode{M}\sim^\rho_\phi\IMode{N}$ and $\IMode{N}\sim^\rho_\phi\IMode{O}$, then
  $\IMode{M}\sim^\rho_\phi\IMode{O}$.
\end{definition}

In other words, to know $\Refines{\Phi}{\phi}{\tau}$ (presupposing
$\RefinesCtx{\Phi}{\Upsilon}$) is to know, for any renaming of $\Upsilon$ to
$\Upsilon'$, which partial equivalence relation on the closed $\tau$-sorted
$\Upsilon'$-terms $\phi$ denotes. At this point, it should be remarked that
this is very similar to the semantical explanation of typehood given in
\cite{martin-lof:1979}, except that we have generalized it to a multi-sorted
setting, and that we have fibred the entire apparatus over collections
$\Upsilon$ of symbols.

\todo[inline]{%
  define maps of refined contexts, then change weakening lemma to
  monotonicity, which is stronger and more useful.
}
\begin{thm}[Weakening]
  If $\Refines{\Phi}{\phi}{\tau}$, then $\Refines{\Phi,u:\psi}{\phi}{\tau}$ (supposing
  $\NotIn{u}{\Dom\Phi}$ and $\Refines{\Phi}{\psi}{\sigma}$).
\end{thm}
\begin{proof}
  This follows trivially from the rule $\RefinesCtxSnoc$ and the canonical renaming
  $\Upsilon\hookrightarrow\Upsilon,u:\sigma$.
\end{proof}

When we have $\Refines{\cdot}{\phi}{\tau}$, we say that $\phi$ \emph{globally}
refines $\tau$ and may simply write $\GRefines{\phi}{\tau}$. By weakening, if
$\GRefines{\phi}{\tau}$ then for all $\Phi$, we have
$\Refines{\Phi}{\phi}{\tau}$ supposing $\RefinesCtx{\Phi}{\Upsilon}$.

\subsubsection{Equality of parametric refinement}

Two refinements are equal when they denote the same PER.\@That is, will define
the refinement equality judgment $\EqRefines{\Phi}{\phi}{\psi}{\tau}$ as
follows, presupposing $\RefinesCtx{\Phi}{\Upsilon}$,
$\Refines{\Phi}{\phi}{\tau}$ and $\Refines{\Phi}{\psi}{\tau}$:
\begin{definition}
  To know $\EqRefines{\Phi}{\phi}{\psi}{\tau}$ is to know, for any substitution
  $\Of{\rho}{\Upsilon\hookrightarrow\OMode{\Upsilon'}}$, the following judgments:
  \begin{gather*}
    \GenJ{
      \MV{m},\MV{n}:\tau
    }{
      \HypJ{
        \IMode{M}\sim^\rho_\phi\IMode{N}
      }{
        \IMode{M}\sim^\rho_\psi\IMode{N}
      }
    }\\
    \GenJ{
      \MV{m},\MV{n}:\tau
    }{
      \HypJ{
        \IMode{M}\sim^\rho_\psi\IMode{N}
      }{
        \IMode{M}\sim^\rho_\phi\IMode{N}
      }
    }
  \end{gather*}

 %% and for any $M, N$
 %% such that $\IsAbt{\cdot}{\Upsilon'}{\cdot}{M}{\tau}$ and
 %% $\IsAbt{\cdot}{\Upsilon'}{\cdot}{N}{\tau}$,
 %% $\IMode{M}\sim^\rho_\phi\IMode{N}$ if and only if
 %% $\IMode{M}\sim^\rho_\psi\IMode{N}$.
\end{definition}

\subsection{Parametric Equality}

The primary judgment concerning refinements is the parametric equality,
$\RMemEq{\Phi}{M}{N}{\phi}$, which presupposes $\RefinesCtx{\Phi}{\Upsilon}$,
$\Refines{\Phi}{\phi}{\tau}$, $\IsAbt{\cdot}{\Upsilon}{\cdot}{M}{\tau}$ and
$\IsAbt{\cdot}{\Upsilon}{\cdot}{N}{\tau}$. The meaning of this judgment is that
$M$ and $N$ are identified by $\phi$ at the identity renaming
$\Of{1}{\IMode{\Upsilon}\hookrightarrow\OMode{\Upsilon}}$:
\[
  \infer{
    \RMemEq{\Phi}{M}{N}{\phi}
  }{
    \IMode{M}\sim^1_\phi\IMode{N}
  }
\]

\subsection{Functional Refinement and Equality}

\newcommand\PRefinesCtxNil{\leftidx{^\parallel}{\!\!\sqsubset}{^\star_{\mathtt{nil}}}}
\newcommand\PRefinesCtxSnoc{\leftidx{^\parallel}{\!\!\sqsubset}{^\star_{\mathtt{snoc}}}}

Next, we define functional refinement
$\HEqRefines{\Phi}{\Psi}{\phi}{\psi}{\tau}$ in terms of parametric refinement,
simultaneously with parametric context refinement
$\PRefinesCtx{\Phi}{\Psi}{\Gamma}$ and functional equality
$\HRMemEq{\Phi}{\Psi}{M}{N}{\phi}$. We will write
$\HRefines{\Phi}{\Psi}{\phi}{\tau}$ as a shorthand for
$\HEqRefines{\Phi}{\Psi}{\phi}{\phi}{\tau}$.

\todo[inline]{%
  Properly express the \emph{syntactic} presuppositions of the sequent
  judgments so that the use of hypothetical and general judgment is
  well-formed.
}

\newcommand\DefPRefinesCtxNil[1]{
  \infer[\PRefinesCtxNil]{
    \PRefinesCtx{#1}{\cdot}{\cdot}
  }{
  }
}
\newcommand\DefPRefinesCtxSnoc[6]{
  \infer[\PRefinesCtxSnoc]{
    \PRefinesCtx{#1}{#2,#3:#4}{#5,#3:#6}
  }{
    \PRefinesCtx{#1}{#2}{#5} &
    \HRefines{#1}{#2}{#4}{#6}
  }
}

Parametric context refinement $\PRefinesCtx{\Phi}{\Psi}{\Gamma}$ shall be
defined as follows, presupposing $\RefinesCtx{\Phi}{\Upsilon}$ and
$\IsVarCtx{\Gamma}$:
\begin{gather*}
  \DefPRefinesCtxNil{\Phi}\qquad
  \DefPRefinesCtxSnoc{\Phi}{\Psi}{x}{\phi}{\Gamma}{\tau}
\end{gather*}

Functional refinement $\HEqRefines{\Phi}{\Psi}{\phi}{\psi}{\tau}$ presupposes
$\PRefinesCtx{\Phi}{\Psi}{\Gamma}$ and $\IsSort{\tau}$, and is explained by
induction on the evidence for its first presupposition.

\paragraph{Case} $\DefPRefinesCtxNil{\Phi}$.
\[
  \infer{
    \HEqRefines{\Phi}{\cdot}{\phi}{\psi}{\tau}
  }{
    \EqRefines{\Phi}{\phi}{\psi}{\tau}
  }
\]

\paragraph{Case} $\DefPRefinesCtxSnoc{\Phi}{\Psi}{x}{\chi}{\Gamma}{\sigma}$,
where $\RefinesCtx{\Phi}{\Upsilon}$.

\begin{gather*}
  \infer{
    \HEqRefines{\Phi}{\Psi,x:\chi}{\phi}{\psi}{\tau}
  }{
    \HPGenJ{\Upsilon}{\Gamma}{\MV{m},\MV{n}:\sigma}{
      \HypJ{
        \HEqRefines{\Phi}{\Psi}{
          \Subst{\MV{m}}{x}{\phi}
        }{
          \Subst{\MV{n}}{x}{\psi}
        }{\tau}
      }{
        \HRMemEq{\Phi}{\Psi}{\MV{n}}{\MV{n}}{\chi}
      }
    }
  }
\end{gather*}


%% Functional equality $\HRMemEq{\Phi}{\Psi}{M}{N}{\phi}$ is defined by
%% induction on the evidence for its presupposition
%% $\PRefinesCtx{\Phi}{\Psi}{\Gamma}$:
%% \begin{gather*}
%%   \infer{
%%     \HRMemEq{\Phi}{\cdot}{M}{N}{\phi}
%%   }{
%%     \RMemEq{\Phi}{M}{N}{\phi}
%%   }\qquad
%%   \infer{
%%     \HRMemEq{\Phi}{\Psi,x:\psi}{M}{N}{\phi}
%%   }{
%%  }
%%\end{gather*}

%%
%% \subsection{Parametric functional equality}
%%
%% Next, we will define the parametric hypothetico-general (or \emph{functional})
%% equality
%% $\IMode{\Phi}\parallel\IMode{\Psi}\vDash\IMode{M}=\IMode{N}\in\IMode{\phi}$
%% simultaneously with variable context well-formedness,
%% $\IMode{\Phi}\parallel\RefinesCtx{\Psi}{\Gamma}$.

\ifdraft{}{
  \bibliographystyle{abbrv}
  \bibliography{refs}
}


\end{document}
